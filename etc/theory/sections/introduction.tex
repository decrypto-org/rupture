\section{Introduction}\label{sec:prev}

Compression and encryption composition has caused serious security problems in
protocols such as TLS \cite{dierks2008tls} over the years with an array of
attacks making major headlines in computer security dissemination forums,
including BREACH \cite{gluck2013breach}, TIME \cite{be2013perfect}, and CRIME
\cite{duong2012crime}. Although the idea that compression is a side-channel that
can be exploited is certainly folklore in the cryptography community and is
documented at least as early as Kelsey's work \cite{kelsey2002compression}, a
proper theoretical model capturing the real-world perspective of these
side-channel attacks is still elusive.

Motivated by this, in this paper we introduce a general model that incorporates
the rendering process that is exploited during these attacks and facilitate a
thorough theoretical analysis of the security of encryption-compression
composition. We showcase the power of our approach by (i) demonstrating an
improved attack framework, called Rupture, that improves substantially the
performance of previous attacks, (ii) showing the above attacks can be expressed
as instances of our compression side-channel attack template, (iii) presenting a
general mitigation strategy, called context hiding, or CTX, that strikes a good balance between
security and efficiency and can be readily incorporated in a TLS deployment
without a significant performance degradation.

% More information on related work can be
%found on Appendix B.

\noindent
\textbf{Our contributions.} First, we implement Rupture,\footnotemark[1] an
open source production-grade framework for conducting compression and encryption
composition attacks. This generic framework provides an extensible mechanism
for experimentally testing attack techniques in a modular setting. We implement
the specific compression attack instance of BREACH. An extensive
description of Rupture's architecture is in section \ref{subsec:rupture}.
This is the first time a working tool is provided for this class of attacks.

We then introduce a model for analyzing compression and encryption composition
attacks, the \textit{adaptive reflection game}, with an accompanying security
definition. This model is novel as it generalizes the concept and the properties
of these attacks. Section \ref{sec:refsec} contains the definition of this
model.

We prove that all encryption functions are insecure when composed with
compression, based on a series of required assumptions. We define the properties
\textit{interdependence} and \textit{compression idealness} and show that these
somewhat innocent properties enable the construction of an attacker, which we
prove breaks the security of the scheme.

Based on our attack framework we achieve significant improvements in our
implementation compared to previous attempts. We improve the complexity from
linear to logarithmic in the secret alphabet's size, we attack block ciphers in
addition to stream ciphers and we employ practical techniques for network level
optimization.

Having established the attack premises, we shift our interest in defending
against such attacks. Although there are several proposed defenses, we feel that
none achieves good compression performance in real-world systems. In this
direction, we propose \textit{context hiding}, a method that mitigates a class
of practical compression side-channel attacks like BREACH.

Our implementation of the context hiding defense technique is called
CTX\footnote[1]{The GitHub repositories of the Rupture and CTX projects make the
source code openly available but have been removed for anonymization purposes
and can be provided upon request.}. CTX runs at the application layer of web
services, is opt-in, and achieves better balance between security and
compression performance compared to existing defense methods.

We experimentally show that CTX performs well in time and space. Specifically,
we find that the performance penalty of CTX compared to other proposed methods
is significantly lower and within accepted rates for real-world applications.

Finally, we provide theoretical arguments as to why CTX is an effective defense
method.
