\section{Related work}\label{sec:related}

\subsection{Compression attacks}

The compression side-channel attack technique was first studied by Kelsey
\cite{kelsey2002compression} in 2002. Kelley and Tamassia\cite{kelley2014secure}
discuss the notion of \textit{entropy-restricted} semantic security. However,
this only models the case where two secrets compress to the same length, making
it difficult to capture practical techniques used in attacks such as BREACH.

Alawatugoda et al. \cite{alawatugoda2015protecting} introduce a model for \textit{Cookie recovery},
\textit{Random cookie indistinguishability}, and \textit{Chosen cookie indistinguishability}.
Nevertheless, their security models are limited to classes of rendering functions
that are very specific. In their models, the attacker is able to directly influence
the output of the rendering function through concatenation.

\subsection{Defenses}

\subsubsection{Disabling compression}\label{subsec:disablecom}
Compression is the basic assumption for the compression detectability, so disabling
compression would break the adaptive reflection game. However, this is not a
viable solution. Disabling compression results in a performance penalty that
outweighs the benefits of the defense.

\subsubsection{SameSite cookies}\label{subsec:samesite}
SameSite cookies \cite{mwest2015firstparty} is a proposed draft for enhancing
web security. The main puprose of this technique is to mitigate cross-site
request forgery attacks. It introduces a new attribute for web cookies, that web
servers may opt-in, which states that cookies may only be included in a
``first-party" context, i.e. in same origin requests. The current BREACH
application depends on cross-origin requests in order to call the reflection
oracle, so enabling SameSite cookies would result in the elimination of this
type of reflection oracle of the Reflection-Security Game.

\subsubsection{Masking secrets}\label{subsec:masking}
Masking is a method for hiding a secret's properties. A mask is a random
bitstream in the secret's size, which is XORed with the secret and concatenated
with the masked output. The secret can then be obtained by applying XOR between
the masked secret and the mask. This method doubles the size of the protected
secret, while also weakening the compression performance. Therefore, it is
applied on high value secrets only, in order to avoid significant performance
penalties. It is used by Facebook \cite{facebookbreach} in order to protect CSRF
tokens against BREACH attacks.
