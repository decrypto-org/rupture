\begin{abstract} When compression is composed with encryption, unexpected
vulnerabilities can arise. Such vulnerabilities have been explored in practical
attacks against TLS such as CRIME, TIME, and BREACH, even after TLS 1.3.
In this work we perform a thorough and systematic study of these attacks
and provide both practical and theoretical contributions.
First, motivated by the lack of production-grade testing tools, we provide Rupture, a
modular scalable open-source attack framework that implements
compression side-channel attacks in a robust manner. Second, we introduce an abstract
theoretical cryptographic model to express such attacks, the \textit{adaptive
reflection security} ``game''. Specifically, we
express \textit{compression idealness} and experimentally show that typical
compression functions exhibit partially ideal behavior. We then argue that
plaintext distributions used in practice follow an \textit{interdependent} joint distribution,
a stronger notion of dependent joint distributions. Based on these two
properties, we prove that \textit{compression-detectability} of predicates
arises, i.e. the ability to compute a predicate of the plaintext using
a reflection attack. Using this we demonstrate that {\em all}
length-preserving encryption schemes are insecure when composed with such
functions. Finally, we propose a novel defense protocol, \textit{context
hiding}, and provide an implementation of it, \textit{CTX}, that effectively
eliminates the threat these attacks pose with minimum performance and
deployment overhead. We argue about the security of our defense method based on
our model, as this technique modifies the plaintext joint distribution in
order to remove interdependence. We also demonstrate that CTX performs
significantly more favorably in terms of compression rate compared to previous
defense techniques.\end{abstract}
