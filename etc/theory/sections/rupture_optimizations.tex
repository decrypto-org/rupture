\subsection{Optimizations}\label{subsec:optimizations}

\subsubsection{Block alignment}\label{subsec:blockalign}
Block ciphers are \textit{length monotonic}, but not \textit{strictly length
monotonic}, as described in section \ref{subsec:lenmonotone}. Specifically, the
length of the encrypted text is rounded up to a product of $\mu$-bits, where
$\mu$ is the block size. This results in plaintext length difference between two
messages not always resulting in length difference between the respective
encrypted ciphertext.

We bypass this problem by using known block alignment techniques
\cite{moller2014poodle}. This method demands issuing multiple requests to the
reflection oracle and including artificial noise. In each request $r_{i, c_j}$
for each candidate $c_j$ in the secret alphabet, we add increasing artificial
noise of specific length. That way, $r_{1, c_j}$ will contain one character
of alignment noise, $r_{2, c_j}$ two characters and so forth. Therefore, for
some alignment noise length $a \in [0, \mu)$ the reflection of the correct
candidate will be $(\delta*\mu)$ and for all incorrect candidates
$(\delta*\mu)+1$. In that case, the incorrect candidates result in one more
block compared to the correct one. This ensures that one out of $\ceil{\mu /
|r_i|}$ requests will result in a block distinction between the alphabet
candidates.

Figure \ref{fig:block_alignment} depicts the block alignment technique
intuitively.

   \begin{figure}[thpb]
      \centering
          \includegraphics[width=0.48\textwidth]{figures/block_alignment.png}
      \caption{The block alignment method}
      \label{fig:block_alignment}
   \end{figure}

\subsubsection{Reflection computation methods}\label{subsec:reflectionmethods}
The reflection strings $r$ should be polynomially computable, as defined in
\ref{subsec:propertycom}. BREACH is parameterized with the secret alphabet
$\Sigma$ and should successfully distinguish a single character of the alphabet
through a series of requests.

The first method of issuing these requests is serial, as described in section
\ref{subsec:rupture}.  The complexity of this attack is $\mathcal{O}(|\Sigma|)$
and the round ends by finding a character of the secret.

The second method of attack is divide and conquer. The alphabet is now divided
into two subsets $\Sigma_1$ and $\Sigma_2 = \Sigma \setminus \Sigma_1$, where
$|\Sigma_1| = |\Sigma_2| = \ceil{|\Sigma| / 2}$. The reflection string for
$\Sigma_1$ consists of $|\Sigma_1|$ substrings separated by an annotation symbol
$\beta$. Each substring is built by concatenatin the known prefix with a
candidate in $\Sigma_1$. For example, if $\Sigma_1$ is $\{``1", ``2"\}$ and the
known prefix is ``abc", using ``-" as $\beta$ the reflection is ``abc1-abc2".
The reflection string for $\Sigma_2$ is constructed similarly.

The end of each round marks the choice of subset $\Sigma_i$ that contains the
correct alphabet symbol and when $|\Sigma_i| = 1$ the attack stage is complete.
Each round reduces the alphabet by half, so the complexity of this attack is
$\mathcal{O}(log|\Sigma|)$. Figure \ref{fig:divide_and_conquer} depicts the
reflection sequence for the case when the alphabet consists of number digits.

   \begin{figure}[thpb]
      \centering
          \includegraphics[width=0.48\textwidth]{figures/divide_and_conquer.png}
      \caption{The divide \& conquer method}
      \label{fig:divide_and_conquer}
   \end{figure}

\subsubsection{Request parallelization}\label{subsec:parallel}
The reflection oracle is generally synchronous, although it may be able to
handle multiple parallel requests from the adversary.  This is the case in
BREACH. Modern web servers are able to handle multiple parallel requests and
browsers can issue a certain amount of parallel requests per domain. This
functionality enables the adversary to issue multiple parallel requests per
candidate and efficiently reduce the execution time of the attack.

\subsubsection{Request soup}
Previous sections demonstrated the need for multiple requests per reflection
string $r_i$. However, communication with the reflection oracle is time
expensive, so it may be preferable to issue multiple requests for a candidate
and treat them as a set rather than separately.

This technique is useful in the case of BREACH. A request set consists of
requests for a symbol $s_i$ in alphabet $\Sigma$. Bigger request sets result in
less time delay and the adversary can measure the mean length over the number of
requests in the set.
