\section{Rupture: An improved attack}\label{subsec:rupture}
Our contributions include the development of a production-grade framework for
implementing this class of attacks called Rupture. Rupture was developed as
part of the work on extending the BREACH attack against modern systems and block
ciphers.

Rupture provides extended functionality regarding the network aspects of the
attack. It offers the ability to inject malicious code in any computer on the
network, as well as traffic inspection and analysis of captured packet lengths.

In addition, Rupture enables the automated computation of the reflection strings
in each round of the attack. For this computation, the adversary should firstly
know a prefix of the secret. This prefix is needed to build the reflection
strings and bootstrap the attack.

Each reflection consists of the known prefix concatenated with a character in
the candidate alphabet. The candidate alphabet initially is equal to the
alphabet of the secret and, as the attack progresses, it may be reduced to a
smaller set of symbols. This construction method ensures that a reflection in
the reflection vector $\bar{r} = <r_1, r_2, ...>$ will match the secret.

The adversary issues consecutive requests, each containing a different
reflection. The goal is to recognize which one matches the secret.

In the case when $r$ matches the secret $s$, the compression will be better that
a case of no-match. Therefore, the network response packets for the matching
reflection will be smaller than the incorrect reflection packets.

The encrypted data pertaining to one response for a single reflection is a
\textit{sample}. The set of samples collected for a particular reflection is a
\textit{sampleset}. The collected samplesets for all candidates in the alphabet
form a \textit{batch}.

The attack is conducted in \textit{rounds}. In each round, a decision is made
about the state of the attack, so either the next byte of the secret becomes
known or the candidate alphabet is drilled down to a smaller set. Rupture
amplifies the results by issuing multiple requests per reflection in each
sampleset in order to achieve better probability of success. Therefore the
analysis uses the collected samplesets, compares the mean length of the
responses for all candidates, and chooses the candidate regarding the smaller
mean response length as the correct one.

This decision is made with some \textit{confidence} which is expressed in bytes.
If the confidence is insufficient, an additional batch is collected and the
analysis is redone using the aggregated lengths from all collected batches until
the confidence is sufficient.

Once enough batches have been collected for a decision to be made with good
confidence, the round of the attack is complete.  Each stage of the attack
consists of multiple rounds and is concluded by decrypting a signle byte of the
secret.

\subsection{Architecture}\label{app:rupture}
Rupture is a service-based architecture framework which contains multiple
independent components. The components are designed to be able to run
on different networks, although easy instances
of the attack can be performed by running all modules on a single system.

The framework assumes a \textit{target} service to be attacked. This target is a
web service which uses TLS and provides HTTPS endpoints. However, this
assumption can be relaxed and attacks against other similar protocols are
possible.  We have designed Rupture to be a good playground for experimentation
for such new attacks. Examples of other encrypted protocols for experimentation
include SMTP and XMPP.

The attack also assumes a user of the target service for which data will be
decrypted, the \textit{victim}. The victim is associated with a particular
target.

There are two underlying assumptions in our attack: The injection and the
sniffing assumptions. These are often, but not necessarily, achieved through the
same means. The injection assumption states that the adversary is able to
inject code to the victim's machine for execution. This code is able to issue
adaptive requests to the target service. Injection in Rupture is implemented by
the \textit{injector} component. The code that is injected is the
\textit{client} component.  The sniffing assumption states that the adversary is
able to observe encrypted network traffic between the victim and the target.
Sniffing is achieved through the \textit{sniffer} component.

The client must issue adaptive requests. For this purpose, it receives commands
through a \textit{command-and-control} channel. These commands are sent to the
client from the \textit{realtime} component with which the client maintains a
persistent connection.

The realtime component is only responsible for communicating with the client.
The actual decisions for the attack are driven by the \textit{backend}, which
maintains a persistent attack state. The backend uses a \textit{database} to
store the state of the attacks and data received from the sniffer to perform
analyses.

The various components are described in detail in the next sections.

\subsubsection{Client}

The client component is Javascript code issues requests towards a chosen
endpoint. It needs to be executed from the browser of the victim that the
adversary is trying to steal secrets from. That way, the victim's authentication
cookies are included in the request and the secrets are included in the
response.

The client contains minimal logic. It connects to the realtime service through a
command-and-control channel and registers itself. It then waits for work
instructions by the command-and-control channel. The client does not take any
decisions or receive data about the progress of the attack other than the
sampleset it is requested to collect. This allows the system to be upgraded
without having to deploy a new client at the victim's network.

As the victim is browsing the Internet, multiple clients will be injected in
insecure pages and run under various contexts. All of them register and maintain
an open connection using web sockets with the realtime service. The realtime service
will activate one of them for the victim while keeping the others dormant. The
activate client will then receive work instructions. If the active client dies, for
example by closing its browser tab, one of the dormant clients will be woken up
to continue the attack.

\subsubsection{Injector}

The injector component is responsible for injecting the client to the victim's
browser. The injection is performed by ARP spoofing (or NDP poisoning in the
case of IPv6) the local network and forwarding all traffic in a
Man-in-the-Middle manner. The fact that all HTTP responses are infected
increases robustness. In order to build the injector module and perform the
above we have used Bettercap.

The injector component runs on the victim's network and is light-weight and
stateless. It can be easily deployed on a small machine and used for massive
attacks. Multiple injectors can be deployed to different networks, all
controlled by the same central command-and-control channel.

\subsubsection{Realtime}

The realtime service is a service which awaits for work requests by clients. It
receives command-and-control connections from various clients which may live on
different networks, orchestrates them, and enforces which will remain dormant
and which ones will receive work, enabling one client per victim as described above.

It maintains open web socket connections with clients and connects to the
backend service using HTTP API calls, facilitating the communication between the
two endpoints. It is built in Node.js \cite{tilkov2010node} and uses the
socket.io framework \cite{rai2013socket} for communication with the clients.

\subsubsection{Sniffer}

The sniffer component is responsible for collecting data from the victim's
network. As the client issues the requests, the sniffer collects the ciphertext
of the request and response packets on the network. This encrypted data is then
transmitted to the backend for further analysis.

The sniffer exposes a HTTP API which is utilized by the backend for controlling
when sniffing starts, when it is completed, and to retrieve the sniffed data. It
is built in Python, using Scapy \cite{biondi2010scapy} for the network functionality.

\subsubsection{Backend}

The backend component controls the attack execution. It initializes the attack
and calculates the samplesets that should be collected. It is responsible for
strategic decision taking, statistical analysis of collected samples, adaptively
advancing the attack, and storing persistent data about the attacks in progress
for future analysis.

It communicates with the realtime component in order to guide it as to
what work should be forwarded to the clients. Meanwhile, it orders the sniffer
to listen and report network traffic and analyzes the captured network data to
calculate the attack confidence. At the end of each
round, information about a single byte of the secret becomes known.

The backend is built in Python, using the Django framework to launch a web
service for communicating with the rest of Rupture's modules.

   \begin{figure}[thpb]
      \centering
          \includegraphics[width=0.48\textwidth]{figures/architecture.png}
      \caption{Rupture Architecture}
   \end{figure}
