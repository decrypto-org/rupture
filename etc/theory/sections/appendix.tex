\section*{Appendix A: Full proofs}

\begin{lemma}[Semantic security]
\end{lemma}

\begin{proof}
    We will prove that if the scheme is not semantically secure, then it is
    necessarily not reflection secure. Assume that the scheme is not
    semantically secure. Then there will exist a PPT adversary $\mathcal{A}$
    such that for all PPT simulators $\mathcal{S}$ we have that $\mathcal{A}$
    has non-negligible advantage to $\mathcal{S}$ in the semantic security game.

    We will now construct an adversary $\mathcal{A'}$ for the reflection
    security game. $\mathcal{A'}$ operates as follows: It makes one query to
    the reflection oracle setting $r_0 = \epsilon$ and receives a response $c_0
    = K_k(s, r_0) = K_k(s) = c$. It then passes that $c$ to the semantic
    security adversary $\mathcal{A}$. It answers all encryption and decryption
    queries of the semantic security adversary by relaying them to its own
    encryption and decryption oracle. Finally, when the semantic security
    adversary outputs $y = y'$, this $y'$ is output by the reflection security
    adversary. Note that $\mathcal{A}$ cannot distinguish whether they are
    playing against the actual semantic security game or being simulated by the
    reflection security adversary, as their view is identical. Therefore:

    \begin{equation}
        Pr[Game_{REF-SEC}^{\mathcal{A'}} = 1] = Pr[Game_{SEM-SEC}^{\mathcal{A}} = 1]
    \end{equation}

    It remains to prove that $\mathcal{A'}$ has significant advantage against
    any reflection simulator $\mathcal{S'}$. Indeed let $\mathcal{S'}$ be any
    reflection game simulator. We will construct a simulator $\mathcal{S}$ for
    the semantic security game. Initially, the simulator $\mathcal{S}$ receives
    the length of the ciphertext $|c|$. They then simulate the reflection game
    simulator $\mathcal{S'}$ as follows. Upon receiving query $r_i$ from the
    reflection game simulator, they answer with $|c| + |r_i|$. When the
    reflection security adversary outputs $y' = y$, this $y$ is output by the
    semantic security simulator. We observe that the reflection game simulator
    $\mathcal{S'}$ cannot distinguish whether they are playing against the
    actual simulated reflection game or being simulated by the semantic
    security simulator. To see this, note that from the length preservation
    assumption we have that $|c| + |r_i| = |K(s, r_i)| = |K(s', r_i)| =
    |K(0^{|s|}, r_i)|$. Therefore:

    \begin{equation}
        Pr[Game_{SEM-SIM}^{\mathcal{S}} = 1] = Pr[Game_{REF-SIM}^{\mathcal{S'}} = 1]
    \end{equation}

    From the assumption that the scheme is not semantically insecure, we know
    that:

    \begin{align*}
        |Pr[Game_{SEM-SEC}^{\mathcal{A}} = 1] - Pr[Game_{SEM-SIM}^{\mathcal{S}} = 1]| =\\
        Adv_{\mathcal{A}, \mathcal{S}} = \text{non-negl}
    \end{align*}

    And therefore, replacing both probabilities with their equals:

    \begin{align*}
        |Pr[Game_{REF-SEC}^{\mathcal{A'}} = 1] -
        Pr[Game_{REF-SIM}^{\mathcal{S'}} = 1]| =\\
        Adv_{\mathcal{A'}, \mathcal{S'}} = \text{non-negl}
    \end{align*}
\end{proof}

\begin{lemma}[Good compression is detectable]
\end{lemma}

\begin{proof}

We make the assumption that a reflection pair $(r_1, r_2)$ is computable.

\begin{align*}
    \Pr[r = r_1|s = s_1] < \Pr[r = r_2|s = s_1]&\land\\
    \Pr[r = r_1|s = s_2] > \Pr[r = r_2|s = s_2]&
\end{align*}

Using the fact that $\mathcal{K}$ is ideal with respect to this distribution,
we deduce that:

\begin{align*}
    |\mathcal{K}(s_1, r_1)| > |\mathcal{K}(s_1, r_2)|&\land\\
    |\mathcal{K}(s_2, r_1)| < |\mathcal{K}(s_2, r_2)|&
\end{align*}

We then define $Q(s)$ to be the predicate "$s = s_1$". This partitions
$\mathcal{M}$ into the distributions $\mathcal{M}_Q$ and
$\mathcal{M}_{\lnot Q}$ both of which are non-empty, as $\mathcal{M}_Q$
contains $s_1$ and $\mathcal{M}_{\lnot Q}$ contains $s_2$. From the fact that
$Q$ is not trivial, we deduce that $\pi < 1$.

Observe, then, that letting $\bar{r} = (r_1, r_2)$ we obtain:

\begin{align*}
    \Pr[cpr^Q_{\mathcal{K}}(s_1, \bar{r}, \mathcal{K}, \mathcal{M})
    \land
    cpr^Q_{\mathcal{K}}(s_2, \bar{r}, \mathcal{K}, \mathcal{M})] = 1
\end{align*}

This completes the proof.

\end{proof}

\begin{lemma}[Compression attack]
\end{lemma}

\begin{proof}

Let $g$ be the boolean function $Q$ on the plaintext. Define the adversary
$\mathcal{A}$ as follows:

\begin{lstlisting}[texcl,mathescape,basicstyle=\small]
def $\mathcal{A}(1^\lambda)$:
    $(r_1, r_2) \leftarrow \mathcal{O}_R(1^\lambda)$

    $l_1 = |\text{Reflect}^{k}_s(r_1)|$
    $l_2 = |\text{Reflect}^{k}_s(r_2)|$

    if $l_1 < l_2$:
        return True
    else:
        return False
\end{lstlisting}

Let $\mathcal{S}$ be an arbitrary simulator. Then we have:
\begin{align*}
    \Pr[\text{Game}_{\text{REF-SIM}}^{\mathcal{SE},\mathcal{S}}
        (\lambda) = 1] &=\\
    \Pr_{x \leftarrow \mathcal{M}, b \leftarrow \mathcal{S}(1^\lambda)}
        [Q(x) = b]
\end{align*}

Letting the random variables $b$ and $x$:
\begin{align*}
    x &\leftarrow \mathcal{M}\\
    b &\leftarrow \mathcal{S}(1^\lambda)
\end{align*}

Due to the independence of the simulator's output with the choice of $x$ we have:
\begin{align*}
    Pr[b = Q(x)] &=\\
    Pr[\lnot b|\lnot Q(x)]Pr[\lnot Q(x)] + Pr[b|Q(x)]Pr[Q(x)] &=\\
    Pr[\lnot b]Pr[\lnot Q(x)] + Pr[b]Pr[Q(x)] &=\\
    (1 - Pr[b])(1 - Pr[Q]) + Pr[b]Pr[Q(x)] &=\\
    1 - Pr[Q(x)] - Pr[b] + 2Pr[b]Pr[Q(x)]
\end{align*}

For a given $\Pr[Q]$, this function is monotonic in $\Pr[b]$ and therefore has
potential extrema at $\Pr[b] = 0$ or $\Pr[b] = 1$, for which cases the function
takes the values $1 - \Pr[Q(x)]$ and $\Pr[Q(x)]$ respectively. Therefore, the
maximum $\Pr[b]$ of the simulator is:
\begin{equation*}
    \Pr[Q(x) = b] = max(\Pr[Q(x)], 1 - \Pr[Q(x)]) = \pi
\end{equation*}

And therefore:
\begin{align*}
    \forall \mathcal{S}:\\
    \Pr[
        \text{Game}_{\text{REF-SIM}}^{\mathcal{SE},\mathcal{S}}
        (\lambda) = 1
    ]
    \leq\\
    max(Pr[Q(x)], 1 - Pr[Q(x)])
\end{align*}

From the compression detectability of Q we know that:
\begin{align*}
    \exists \alpha \text{ non-negl}:\\
    \Pr_{s_1 \leftarrow \mathcal{M}_Q,
         s_2 \leftarrow \mathcal{M}_{\lnot Q}}
         [cpr^Q_{\kappa}(s_1, \overbar{r}) \land
          cpr^Q_{\kappa}(s_2, \overbar{r})]
    \geq\\
    \pi + \alpha(\lambda)
\end{align*}

Therefore:
\begin{align*}
    \Pr_{s_1 \leftarrow \mathcal{M}_Q}
         [cpr^Q_{\kappa}(s_1, \overbar{r})]
    \geq
    \pi + \alpha(\lambda) \land\\
    \Pr_{s_2 \leftarrow \mathcal{M}_{\lnot Q}}
         [cpr^Q_{\kappa}(s_2, \overbar{r})]
    \geq
    \pi + \alpha(\lambda)
\end{align*}

And so:
\begin{align*}
    \Pr_{s \leftarrow \mathcal{M}}
         [cpr^Q_{\kappa}(s, \overbar{r})]
    =\\
    \Pr[cpr^Q_{\kappa}(s, \overbar{r})|Q(s)]\Pr[Q(s)]
    +\\
    \Pr[cpr^Q_{\kappa}(s, \overbar{r})|\lnot Q(s)]\Pr[\lnot Q(s)]
    \geq\\
    (\pi + \alpha(\lambda))(\Pr[Q(s)] + (1 - \Pr[Q(s)))
    =\\
    \pi + \alpha(\lambda)
\end{align*}

Let us now examine the event of $\mathcal{A}$ being successful, denoted Succ,
when $cpr^Q_{\kappa}(s, \overbar{r})$. Assuming $Q(s)$:
\begin{align*}
    \Pr[|\mathcal{K}(s, r_1)| < |\mathcal{K}(s, r_2)||Q(s)]
    = \pi + \alpha(\lambda)
\end{align*}

And from the strict length monotonicity of $\textrm{Enc}$ it follows that:
\begin{align*}
    \Pr[&|\textrm{Enc}(\mathcal{K}(s, r_1))| <\\&|\textrm{Enc}(\mathcal{K}(s, r_2))||Q(s)]
        \geq \pi + \alpha(\lambda)\\
    \Rightarrow \Pr[&
        |\text{Reflect}^{k}_s(r_1)|
        <
        |\text{Reflect}^{k}_s(r_2)||Q(s)
    ]
        \geq \pi + \alpha(\lambda)\\
    \Rightarrow \Pr[&l_1 < l_2|Q(s)]
        \geq \pi + \alpha(\lambda)\\
    \Rightarrow \Pr[&\text{Succ}|Q(s)]
        \geq \pi + \alpha(\lambda)\\
\end{align*}

The case for $\lnot Q(s)$ is the same, but with a different inequality direction:
\begin{align*}
    \Pr[|\mathcal{K}(s, r_1)| > |\mathcal{K}(s, r_2)||\lnot Q(s)]
        \geq\\
        \pi + \alpha(\lambda)\\
    \Rightarrow \Pr[\text{Succ}|\lnot Q(s)] \geq\\
    \pi + \alpha(\lambda)\\
\end{align*}

And so the probability of success is given:
\begin{align*}
    \Pr[\text{Succ}] =\\
    \Pr[\text{Succ}|Q(s)]\Pr[Q(s)]
    +
    \Pr[\text{Succ}|\lnot Q(s)]\Pr[\lnot Q(s)] \geq\\
    \pi + \alpha(\lambda)
\end{align*}

Therefore,
\begin{align*}
    \forall PPT \mathcal{S}:
    \text{Adv}_{\mathcal{SE}(\textrm{Enc}, \textrm{Com}), \mathcal{A}, \mathcal{S}}
        (1^\lambda)
    \geq\\
    |\pi + \alpha(\lambda) - \pi| = \alpha(\lambda)
\end{align*}

Which is non-negligible.

\end{proof}

\begin{lemma}[Amplification]
\end{lemma}

\begin{proof}

From the fact that $Q$ is compression-detectable and from the Compression Attack Theorem, we have that:
\begin{align*}
    \Pr_{s \leftarrow \mathcal{M}}
         [cpr^Q_{\kappa}(s, \overbar{r})]
    =\\
    \pi + \alpha(\lambda)
\end{align*}

Some elements $s \in \mathcal{M}$ allow for better compression-detectability than others under the fixed
reflection vector $\overbar{r}$. Call these elements \textit{amplifiable} and define predicate:
\begin{align*}
    Amp(s) \defeq
    \Pr[cpr^Q_{\kappa}
     (s, \overbar{r})]
    \geq
    \frac{1}{2} + \frac{\alpha(\lambda)}{2}
\end{align*}

We will now obtain a lower bound on the probability of an element being amplifiable.

Let:

% B derivation: (where \beta(\lambda) = \alpha(\lambda) / 2)
% B + (\frac{1}{2} + \alpha(\lambda)/2)(1 - B) = \pi + \alpha(\lambda)
% B + \frac{1}{2} + \alpha(\lambda)/2 - B(\frac{1}{2} + \beta(\lambda)) = \pi + \alpha(\lambda)
% \frac{1}{2} + \beta(\lambda) + B(\frac{1}{2} - \beta(\lambda)) = \pi + \alpha(\lambda)
% B(\frac{1}{2} - \beta(\lambda)) = \pi + \alpha(\lambda) - \frac{1}{2} - \beta(\lambda)
% B = (\pi + \alpha(\lambda) - \frac{1}{2} - \beta(\lambda)) / (\frac{1}{2} - \beta(\lambda))
% B = (\pi + \alpha(\lambda) - \frac{1}{2} - \frac{\alpha(\lambda)}{2}) / (\frac{1}{2} - \frac{\alpha(\lambda)}{2})
% B = (\pi - \frac{1}{2} + \frac{\alpha(\lambda)}{2}) / (\frac{1}{2} - \frac{\alpha(\lambda)}{2})
% B = \pi / (\frac{1}{2} - \frac{\alpha(\lambda)}{2}) - (\frac{1}{2} - \frac{\alpha(\lambda)}{2}) / (\frac{1}{2} - \frac{\alpha(\lambda)}{2})
% B = \pi / (\frac{1}{2} - \frac{\alpha(\lambda)}{2}) - 1
\begin{align*}
    B = \frac{\pi}{\frac{1}{2} - \frac{\alpha(\lambda)}{2}} - 1
\end{align*}

$B$ is non-negligible in $\lambda$.

Assume, for the sake of contradiction, that:
\begin{align*}
    \Pr_{s \leftarrow \mathcal{M}}
    [Amp(s)] < B
\end{align*}

Then we have:
\begin{align*}
    \Pr_{s \leftarrow \mathcal{M}}
         [cpr^Q_{\kappa}(s, \overbar{r})]
    =\\
    \Pr_{s \leftarrow \mathcal{M}}
         [cpr^Q_{\kappa}(s, \overbar{r})|Amp(s)]\Pr[Amp(s)]
    +\\
    \Pr_{s \leftarrow \mathcal{M}}
         [cpr^Q_{\kappa}(s, \overbar{r})|\lnot Amp(s)]\Pr[\lnot Amp(s)]
    <\\
    B + (\frac{1}{2} + \frac{\alpha}{2})(1 - B)
\end{align*}

But then:
\begin{align*}
    B + (\frac{1}{2} + \frac{\alpha(\lambda)}{2})(1 - B) =
    \pi + \alpha(\lambda)
\end{align*}

And this contradicts the assumption that $Q$ is compression-detectable. Therefore:
\begin{align*}
    \Pr_{s \leftarrow \mathcal{M}}
    [Amp(s)] \geq B
\end{align*}

It remains to show that the advantage of the adversary can be
arbitrarily large, i.e. that for some negligible $C$ we have:
\begin{align*}
    \text{Adv}_{\mathcal{SE}(\textrm{Enc}, \textrm{Com}), \mathcal{A}, \mathcal{S}_{Amp}}
    (1^\lambda) = 1 - \pi - C
\end{align*}

Indeed, observe that the amplifying adversary performs a repeated Bernoulli
trial with $k$ repetitions and extracts a majority. Let $X$ be the number of
repetitions that are successful for the adversary. $X$ is defined:
\begin{align*}
    X \defeq |\{ i: cpr^Q_{\kappa}(s, \overbar{r}) \}|
\end{align*}

$X$ follows the binomial distribution, and therefore its expected value is:
\begin{align*}
    E[X] = \frac{k}{2} + \frac{\alpha k}{2}
\end{align*}

Let Succ denote the event of the amplified adversary succeeding. Then Succ
is equivalent to $X > \frac{k}{2}$.

Because $\alpha$ is non-negligible and due to the tail bounds of the binomial
distribution, we have:
\begin{align*}
    Pr[Succ] = 1 - C(k)
\end{align*}

Where $C$ is a negligible function.
\end{proof}
